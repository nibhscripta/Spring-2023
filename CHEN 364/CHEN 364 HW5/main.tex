\documentclass[12pt]{article}
\usepackage{amsmath}
\usepackage{graphicx}
\usepackage{hyperref}
\usepackage[utf8]{inputenc}
\usepackage{geometry}
\usepackage{mathtools}
\usepackage{empheq}
\usepackage{listings}
\usepackage{xcolor}
\usepackage{minted}
\usepackage{svg}
\definecolor{LightGray}{gray}{0.9}

\graphicspath{ {./assets/} }
\geometry{margin=0.75in}

\title{CHEN 364 HW 3}
\author{Mark Levchenko}
\date{February 2023}

\begin{document}

\begin{enumerate}
 
% Problem 1 %%%%%%%%%%%%%%%%%%%%%%%%%%%%%%%%%%%%%%%
\newpage
    \item Problem 1
    \[
        \mathrm{A} + \mathrm{B} \rightarrow \mathrm{C}  
    \]
    \begin{align*}
        \intertext{Reactor 1 mole balances}
        \frac{dN_{A,1}}{dt} &= -r_{A,1} V_1 - \frac{F N_{A,1}}{V_1} \\
        \frac{dN_{C,1}}{dt} &= r_{A,1} V_1 - \frac{F N_{C,1}}{V_1} \\
        V_1 &= V_{1,0} - F t \\
        N_A &= N_B \text{ equimolar initial composition} \\
        \intertext{Reactor 2 mole balances}
        \frac{dN_{A,2}}{dt} &= -r_{A,2} V_2 + \frac{F N_{A,1}}{V_1} \\
        \frac{dN_{C,2}}{dt} &= r_{A,2} V_2 + \frac{F N_{C,1}}{V_1} \\
        V_2 &= F t \\
        \intertext{Reactions}
        r_{A,1} = k C_A C_B &= k \frac{N_{A,1}^2}{V_1^2} \\ 
        r_{A,2} &= k \frac{N_{A,2}^2}{V_2^2} \\ 
        \intertext{Solve the system of ODEs with the following paramters}
        & k = 0.0445 \text{ L/mol/min} \\
        & F = 10 \text{ L/min} \\
        & V_{1,0} = 1000 \text{ L} \\
        & C_{A0} = 2 \text{ M} \\
        & C_A = \frac{N_A}{V}
    \end{align*}

    Reactor 1 plot:

    \begin{center}
        \includesvg[width=0.9\textwidth]{assets/p1_r1.svg}
    \end{center}

    Reactor 2 plot:

    \begin{center}
        \includesvg[width=0.9\textwidth]{assets/p1_r2.svg}
    \end{center}

    Code:

\begin{minted}[
framesep=2mm,
baselinestretch=1.2,
bgcolor=LightGray,
fontsize=\footnotesize,
breaklines,
]{python}
import numpy as np
import matplotlib.pyplot as plt
from scipy.integrate import solve_ivp

def ode(t, y):
    f = y*0

    N_A1 = y[0]
    N_C1 = y[1]
    N_A2 = y[2]
    N_C2 = y[3]

    k = 0.0445
    F = 10
    V_1_0 = 1000

    V_1 = V_1_0 - F * t
    V_2 = F * t

    r_A1 = k * N_A1**2 / V_1**2
        
    if V_2 == 0:
        r_A2 = 0
    else:
        r_A2 = k * N_A2**2 / V_2**2

    f[0] = -r_A1 * V_1 - F * N_A1 / V_1
    f[1] = r_A1 * V_1 - F * N_C1 / V_1
    f[2] = -r_A2 * V_2 + F * N_A1 / V_1
    f[3] = r_A2 * V_2 + F * N_C1 / V_1

    return f

V_1_0 = 1000.
C_A1_0 = 2.
N_A1_0 = V_1_0 *  C_A1_0


init_cond = [N_A1_0, 0., 0., 0.]

ode_kwargs = {
    'method': 'Radau',
    'atol': 1e-8,
    'rtol': 1e-8,
}

sol = solve_ivp(ode, [0.001, 99.999], init_cond, **ode_kwargs)

F = 10
V_1 = V_1_0 - F * sol.t
V_2 = F * sol.t

C_A1 = sol.y[0] / V_1
C_C1 = sol.y[1] / V_1
C_A2 = sol.y[2] / V_2
C_C2 = sol.y[3] / V_2

plt.plot(sol.t, C_A1, label=r"$C_{A1}/C_{B1}$")
plt.plot(sol.t, C_C1, label=r"$C_{C1}$")
plt.xlabel(r"$t$ (min)")
plt.ylabel(r"$C$ (M)")
plt.title("Problem 1 Reactor 1 Concentrations")
plt.legend()
plt.show()


plt.plot(sol.t, C_A2, label=r"$C_{A2}/C_{B2}$")
plt.plot(sol.t, C_C1, label=r"$C_{C2}$")
plt.xlabel(r"$t$ (min)")
plt.ylabel(r"$C$ (M)")
plt.title("Problem 1 Reactor 2 Concentrations")
plt.legend()
plt.show()
\end{minted}
% Problem 2 %%%%%%%%%%%%%%%%%%%%%%%%%%%%%%%%%%%%%%%
\newpage
    \item Problem 2
    
    Assume Constant volume

    \begin{enumerate}
        \item 
        \begin{align*}
            \intertext{Mole balances}
            \frac{dC_A}{dt} &= -r_1 \\
            \frac{dC_B}{dt} &= r_1 - r_2 \\
            \frac{dC_C}{dt} &= r_2 \\
            r_1 &= k_1 C_A \\
            r_2 &= k_2 C_B \\
            \intertext{Solve the system of ODEs with the following paramters}
            & k_1 = 0.4 \text{ h$^{-1}$} \\
            & k_2 = 0.01 \text{ h$^{-1}$} 
        \end{align*}

        Concentration plot:

        \begin{center}
            \includesvg[width=0.8\textwidth]{assets/p2_a.svg}
        \end{center}

        \item 
        \begin{align*}
            \intertext{Mole balances}
            \frac{dC_A}{dt} &= -r_1 + r_{-1} \\
            \frac{dC_B}{dt} &= r_1 - r_2 - r_{-1} \\
            \frac{dC_C}{dt} &= r_2 \\
            r_1 &= k_1 C_A \\
            r_2 &= k_2 C_B \\
            r_{-1} &= k_{-1} C_B \\
            \intertext{Solve the system of ODEs with the following paramters}
            & k_1 = 0.4 \text{ h$^{-1}$} \\
            & k_2 = 0.01 \text{ h$^{-1}$} \\
            & k_{-1} = 0.3 \text{ h$^{-1}$}
        \end{align*}

        Concentration plot:

        \begin{center}
            \includesvg[width=0.8\textwidth]{assets/p2_b.svg}
        \end{center}

        \item 
        \begin{align*}
            \intertext{Mole balances}
            \frac{dC_A}{dt} &= -r_1 + r_{-1} \\
            \frac{dC_B}{dt} &= r_1 - r_2 - r_{-1} + r_{-2} \\
            \frac{dC_C}{dt} &= r_2 - r_{-2}\\
            r_1 &= k_1 C_A \\
            r_2 &= k_2 C_B \\
            r_{-1} &= k_{-1} C_B \\
            r_{-2} &= k_{-2} C_C \\
            \intertext{Solve the system of ODEs with the following paramters}
            & k_1 = 0.4 \text{ h$^{-1}$} \\
            & k_2 = 0.01 \text{ h$^{-1}$} \\
            & k_{-1} = 0.3 \text{ h$^{-1}$} \\
            & k_{-2} = 0.005 \text{ h$^{-1}$}
        \end{align*}

        Concentration plot:

        \begin{center}
            \includesvg[width=0.8\textwidth]{assets/p2_c.svg}
        \end{center}
    \end{enumerate}

    Code for all three parts:

\begin{minted}[
framesep=2mm,
baselinestretch=1.2,
bgcolor=LightGray,
fontsize=\footnotesize,
breaklines,
]{python}
import numpy as np
import matplotlib.pyplot as plt

from scipy.integrate import solve_ivp

ode_kwargs = {
    'method': 'Radau',
    'atol': 1e-8,
    'rtol': 1e-8,
}

'''
####################
#      Part A      #
####################
'''

def ode_a(t, y):
    f = y*0

    C_A = y[0]
    C_B = y[1]
    C_C = y[2]

    k_1 = 0.4
    k_2 = 0.01
    k_b1 = 0.3
    k_b2 = 0.005

    r_1 = k_1 * C_A
    r_2 = k_2 * C_B

    f[0] = -r_1
    f[1] = r_1 - r_2
    f[2] = r_2

    return f

C_A0 = 1.6

init_cond = [C_A0, 0, 0]

sol_a = solve_ivp(ode_a, [0, 20], init_cond, **ode_kwargs)

plt.plot(sol_a.t, sol_a.y[0], label=r"$C_{A}$")
plt.plot(sol_a.t, sol_a.y[1], label=r"$C_{B}$")
plt.plot(sol_a.t, sol_a.y[2], label=r"$C_{C}$")
plt.title("Problem 2 Part a Concentration Plot")
plt.xlabel(r"$t$ (hr)")
plt.ylabel("Concentration (mol/L)")
plt.legend()
plt.show()

'''
####################
#      Part B      #
####################
'''

def ode_b(t, y):
    f = y*0

    C_A = y[0]
    C_B = y[1]
    C_C = y[2]

    k_1 = 0.4
    k_2 = 0.01
    k_b1 = 0.3
    k_b2 = 0.005

    r_1 = k_1 * C_A
    r_2 = k_2 * C_B
    r_b1 = k_b1 * C_B

    f[0] = -r_1 + r_b1
    f[1] = r_1 - r_2 - r_b1
    f[2] = r_2

    return f

C_A0 = 1.6

init_cond = [C_A0, 0, 0]

sol_b = solve_ivp(ode_b, [0, 20], init_cond, **ode_kwargs)

plt.plot(sol_b.t, sol_b.y[0], label=r"$C_{A}$")
plt.plot(sol_b.t, sol_b.y[1], label=r"$C_{B}$")
plt.plot(sol_b.t, sol_b.y[2], label=r"$C_{C}$")
plt.title("Problem 2 Part b Concentration Plot")
plt.xlabel(r"$t$ (hr)")
plt.ylabel("Concentration (mol/L)")
plt.legend()
plt.show()

'''
####################
#      Part C      #
####################
'''

def ode_c(t, y):
    f = y*0

    C_A = y[0]
    C_B = y[1]
    C_C = y[2]

    k_1 = 0.4
    k_2 = 0.01
    k_b1 = 0.3
    k_b2 = 0.005

    r_1 = k_1 * C_A
    r_2 = k_2 * C_B
    r_b1 = k_b1 * C_B
    r_b2 = k_b2 * C_C

    f[0] = -r_1 + r_b1
    f[1] = r_1 - r_2 - r_b1 + r_b2
    f[2] = r_2 - r_b2

    return f

C_A0 = 1.6

init_cond = [C_A0, 0, 0]

sol_c = solve_ivp(ode_c, [0, 20], init_cond, **ode_kwargs)

plt.plot(sol_c.t, sol_c.y[0], label=r"$C_{A}$")
plt.plot(sol_c.t, sol_c.y[1], label=r"$C_{B}$")
plt.plot(sol_c.t, sol_c.y[2], label=r"$C_{C}$")
plt.title("Problem 2 Part c Concentration Plot")
plt.xlabel(r"$t$ (hr)")
plt.ylabel("Concentration (mol/L)")
plt.legend()
plt.show()
\end{minted}

% Problem 3 %%%%%%%%%%%%%%%%%%%%%%%%%%%%%%%%%%%%%%%
\newpage
    \item Problem 3
    \begin{align*}
        S_{B/X} &= \frac{r_B}{r_X} \\
        S_{B/Y} &= \frac{r_B}{r_Y} \\
        S_{B/XY} &= \frac{r_B}{r_X + r_Y} \\
        r_X &= k_1 C_A^{1/2} \\
        r_B &= k_2 C_A \\
        r_Y &= k_3 C_A^2 \\
        & k_1 = 0.004 \text{ (mol/L)$^{1/2} \cdot$min$^{-1}$} \\
        & k_2 = 0.3 \text{ min$^{-1}$} \\
        & k_3 = 0.25 \text{ L/mol $\cdot$ min}
    \end{align*}

    Plot and determine the concentration at maximum selectivity.

    \begin{center}
        \includesvg[width=0.7\textwidth]{assets/p3.svg}
    \end{center}

    \begin{align*}
        \Aboxed{C_{A,opt} &= 0.04 \text{ mol/L} }\\
        V &= \frac{F \left(C_{A0} - C_{A,opt}\right)}{-r_A} \\
        V &= \frac{F \left(C_{A0} - C_{A,opt}\right)}{k_1 C_{A,opt}^{1/2} + k_2 C_{A,opt} + k_3 C_{A,opt}^2} \\
        C_{A0} &= \frac{P_0}{R T_0} \\
        V &= \frac{F \left(\frac{P_0}{R T_0} - C_{A,opt}\right)}{k_1 C_{A,opt}^{1/2} + k_2 C_{A,opt} + k_3 C_{A,opt}^2} \\
        & F = 10 \\
        & T = 300 \\
        & P = 4 \\
        V &= \frac{10 \left(\frac{4}{0.08206 \cdot 300} - 0.04\right)}{0.004 \cdot 0.04^{1/2} + 0.3 \cdot 0.04 + 0.25 \cdot 0.04^2} \\
        \Aboxed{V &= 92.8 \text{ L}}
    \end{align*}

    Code:

\begin{minted}[
framesep=2mm,
baselinestretch=1.2,
bgcolor=LightGray,
fontsize=\footnotesize,
breaklines,
]{python}
import numpy as np
import matplotlib.pyplot as plt

C_A_ran = np.linspace(0.01, 0.2, 10000)

k_1 = 0.004
k_2 = 0.3
k_3 = 0.25

SX = lambda C_A: k_2 * C_A / (k_1 * C_A**0.5)
SY = lambda C_A: k_2 * C_A / (k_3 * C_A**2)
SXY = lambda C_A: k_2 * C_A / (k_1 * np.sqrt(C_A) + k_3 * C_A**2)

fig, ax = plt.subplots(3, figsize=(6.4, 2*4.8))
ax[0].plot(C_A_ran, SX(C_A_ran), label="Reaction X Selectivity")
ax[0].set_ylabel(r"$S_{B/X}$")
ax[1].plot(C_A_ran, SY(C_A_ran), "tab:orange", label="Reaction Y Selectivity")
ax[1].set_ylabel(r"$S_{B/Y}$")
ax[2].plot(C_A_ran, SXY(C_A_ran), "tab:green", label="Reaction X & Y Selectivity")
ax[2].set_ylabel(r"$S_{B/XY}$")
ax[2].set_xlabel(r"$C_A$ (mol/L)")
ax[0].set_title("Problem 3 Selectivity Plots")
fig.legend()
plt.show()

C_A_opt_index = np.where(SXY(C_A_ran)==SXY(C_A_ran).max())[0][0]
C_A_opt = C_A_ran[C_A_opt_index]

print(f"Optimum C_A = {C_A_opt}")

F = 10 
C_A0 = 4 / 0.08206 / (27 + 273.15)
print(f"CSTR volume = {F * (C_A0 - C_A_opt) / (k_1 * C_A_opt**0.5 + k_2 * C_A_opt + k_3 * C_A_opt**2)}")
\end{minted}
    
% Problem 4 %%%%%%%%%%%%%%%%%%%%%%%%%%%%%%%%%%%%%%%
\newpage
    \item Problem 4
    
    \[
        \mathrm{A} + \mathrm{B} \rightarrow \mathrm{C} + \mathrm{D} 
    \]
    \[
        \mathrm{C} + \mathrm{D} \rightarrow \mathrm{U} 
    \]

    A membrane that selectively removes C would cause the concentration of C in the reactor to be very low or nearly zero. Because C participates in the undersired reaction, a low concentration of C in the reactor will lead the undersired reaction to be slow, and thus very little of the undersired product would be produced.

    Select a memebrane reactor that selectively allows C to leave.
    

\end{enumerate}


\end{document}