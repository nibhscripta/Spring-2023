\documentclass[12pt]{article}
\usepackage{amsmath}
\usepackage{graphicx}
\usepackage{hyperref}
\usepackage[utf8]{inputenc}
\usepackage{geometry}
\usepackage{mathtools}
\usepackage{empheq}
\usepackage{listings}
\usepackage{xcolor}
\usepackage{minted}
\usepackage{svg}


\definecolor{LightGray}{gray}{0.9}

\graphicspath{ {./assets/} }
\geometry{margin=0.5in}

\title{CHEN 461 HW 1}
\author{Mark Levchenko}
\date{January 2023}

\begin{document}


\begin{enumerate}

% Problem 1 %%%%%%%%%%%%%%%%%%%%%%%%%%%%%%%%%%%%%%%%%%%%%%%%%%%%%%%%%%%%%%%%%%%%%%%%%%%
\newpage
\item Problem 3.8
    \begin{align*}
        \tau \frac{dy}{dt} + y(t) &= k u(t) \\
        u(t) &= t\mathcal{H}(t) - t\mathcal{H}(t - 2) + (-t + 4)\mathcal{H}(t-2) - (-t + 4)\mathcal{H}(t-4) \\
        u(t) &= t\mathcal{H}(t) - 2(t - 2)\mathcal{H}(t - 2) + (t - 4)\mathcal{H}(t-4) \\
        0.5 \cdot \frac{dy}{dt} + y(t) &= t\mathcal{H}(t) - 2(t - 2)\mathcal{H}(t - 2) + (t - 4)\mathcal{H}(t-4) \\
        \mathcal{L}\left\{ 0.5 \cdot \frac{dy}{dt} + y(t) \right\} &= \mathcal{L}\left\{ t\mathcal{H}(t) - 2(t - 2)\mathcal{H}(t - 2) + (t - 4)\mathcal{H}(t-4) \right\} \\
        0.5 s Y(s) + Y(s) &= \frac{1}{s^2} - \frac{2 e^{-2s}}{s^2} + \frac{e^{-4s}}{s^2} \\
        Y(s) \left(\frac{s}{2} + 1\right) &= \frac{1}{s^2} - \frac{2 e^{-2s}}{s^2} + \frac{e^{-4s}}{s^2} \\
        \frac{1}{\left(\frac{s}{2} + 1\right) s^2} &= \frac{A}{s} + \frac{B}{s^2} + \frac{C}{\frac{s}{2} + 1} \\
        1 &= A \left(\frac{s^2}{2} + s\right) + B \left(\frac{s}{2} + 1\right) + Cs^2 \\
        \frac{A}{2} + C &= 0 \\
        A + \frac{B}{2} &= 0 \\
        B &= 1 \\
        A &= -\frac{1}{2} \\
        C &= -\frac{1}{4} \\
        Y(s) &= \frac{1}{2} \left(\frac{2}{s^2} - \frac{1}{s + 2} - \frac{1}{s}\right) \\
        &- \frac{1}{2} \left(\frac{2}{s^2} - \frac{1}{s + 2} - \frac{1}{s}\right) \left( 2 e^{-2s} \right) \\
        &+ \frac{1}{2} \left(\frac{2}{s^2} - \frac{1}{s + 2} - \frac{1}{s}\right) \left( e^{-4s} \right)
    \end{align*}

    Take the inverse Laplace Transform of the above equation.
    
    \[
        \boxed{
        y(t) = \frac{1}{2} \left[ 
                                    {\left( 2t - 9 + e^{8 - 2 t}\right) \mathcal{H}(t - 4)} 
                                    - {2 \left( 2 t - 5 + e^{4 - 2 t}\right) \mathcal{H}(t - 2)} 
                                    + {\left( 2 t - 1 + e^{-2 t}\right) \mathcal{H}(t)} 
                            \right]
        }
    \]

    Solution plot:

    \includesvg{CHEN461_HW3_3.8_fig.svg}

    Code to produce the output plot:

\begin{minted}[
framesep=2mm,
baselinestretch=1.2,
bgcolor=LightGray,
fontsize=\footnotesize,
breaklines,
]{python}
import numpy as np
import matplotlib.pyplot as plt
from sympy.integrals import inverse_laplace_transform, laplace_transform
from sympy.abc import t, s
from sympy.functions import exp, Heaviside
from sympy import lambdify
# input a sympy expression
u_expr = t*Heaviside(t) - 2*(t-2)*Heaviside(t-2) + (t-4)*Heaviside(t-4)
# convert sympy expression u(t) into lambda func
u = lambdify(t, u_expr, 'numpy')
# lambdify won't work on Heaviside() with sympy version <1.11
# inverse laplace transform Y(s)
y_expr = inverse_laplace_transform( (1 - 2*exp(-2*s) + exp(-4*s)) / s**2 / (s/2 + 1), s, t)
# convert sympy expression y(t) into lambda func
y = lambdify(t, y_expr, 'numpy')
t_range = np.linspace(0, 6, 100)
# plotting
plt.plot(t_range, y(t_range))
plt.plot(t_range, u(t_range), '--')
plt.xlabel(r'$t$')
plt.ylabel(r'$y(t)$')
plt.title(r'Problem 3.8 Response Plot')
plt.legend(['Response', 'Input'])
\end{minted}



% Problem 2 %%%%%%%%%%%%%%%%%%%%%%%%%%%%%%%%%%%%%%%%%%%%%%%%%%%%%%%%%%%%%%%%%%%%%%%%%%%
\newpage
\item Problem 3.9
    \begin{align*}
        \intertext{Response:}
        y(t) &= \frac{k M}{\epsilon} \left( 1 - e^{\frac{-t}{\tau}} \right) \\
        \intertext{Assume the sensor is perfectly calibrated:}
        k &= 1 \\
        \epsilon &= 1 \mathrm{min} \\
        \intertext{At $t=$1 min:}
        T &\approx 39^\circ\mathrm{C} \\
        T_0 &= 20 \\
        \frac{M}{\epsilon} &= 50 - 20 = 30 \\
        T &= 19 \\
        19 &= 30 \cdot \left( 1 - e^{\frac{-1}{\tau}} \right) \\
        \Aboxed{\tau &= 0.997 \mathrm{min}}
    \end{align*}




% Problem 3 %%%%%%%%%%%%%%%%%%%%%%%%%%%%%%%%%%%%%%%%%%%%%%%%%%%%%%%%%%%%%%%%%%%%%%%%%%%
\newpage
\item Problem 3.10
    \begin{align*}  
        \intertext{Input:}
        T_L(t) &= 20 + \sin{(2t)} \\
        \omega &= 2 \\
        M &= 1 \\
        \text{output amplitude} &= \frac{kM}{\sqrt{1 + (\omega \tau)^2}} \\
        \text{output amplitude} &= 0.1 \text{ from plot} \\
        \intertext{Assume the sensor is perfectly calibrated:}
        k &= 1 \\
        0.1 &= \frac{1}{\sqrt{1 + (2\tau)^2}} \\
        \intertext{Solve for $\tau$:}
        \Aboxed{\tau &= 4.975}
    \end{align*}


% Problem 4 %%%%%%%%%%%%%%%%%%%%%%%%%%%%%%%%%%%%%%%%%%%%%%%%%%%%%%%%%%%%%%%%%%%%%%%%%%%
\newpage
\item Problem 3.12
    \begin{align*}
        \tau \frac{dT}{dt} + T(t) &= k u(t) \\
        \tau &= 10\mathrm{s} = \frac{10\mathrm{s}}{60\mathrm{s/min}} = \frac{1}{6}\mathrm{min} \\
        \intertext{Assume the sensor is perfectly calibrated:}
        k &= 1 \\
        u(t) &= 20 + 5e^{-t} \\
        \intertext{Make $T_0 = 20^\circ$C:}
        T(0) &= 0 \text{, 0 initial condition} \\
        u(t) &= 5e^{-t} \\
        \frac{1}{6} \frac{dT}{dt} + T(t) &= 5e^{-t} \\
        \mathcal{L}\left\{ \frac{1}{6} \frac{dT}{dt} + T(t) \right\} &= \mathcal{L}\left\{ 5e^{-t} \right\} \\
        \frac{s T(s)}{6} + T(s) &= \frac{5}{s + 1} \\
        T(s) &= \frac{5}{(s + 1) (\frac{s}{6} + 1)} \\
        \frac{5}{(s + 1) (\frac{s}{6} + 1)} &= \frac{A}{s+1} + \frac{B}{\frac{s}{6} + 1} \\
        5 &= A \left( \frac{s}{6} + 1 \right) + B (s + 1) \\
        \frac{A}{6} + B &= 0 \\
        A + B &= 5 \\
        A &= 6 \\
        B &= -1 \\ 
        T(s) &= \frac{6}{s+1} + \frac{-1}{\frac{s}{6} + 1} \\
        \mathcal{L}^{-1}\left\{ T(s) \right\} &= \mathcal{L}^{-1}\left\{ \frac{6}{s+1} - \frac{6}{s + 6} \right\} \\
        T(t) &= 6 \left(e^{- t} - e^{- 6 t} \right) 
    \end{align*}

    Solution plot:

    \includesvg{CHEN461_HW3_3.12_fig.svg}

\begin{minted}[
framesep=2mm,
baselinestretch=1.2,
bgcolor=LightGray,
fontsize=\footnotesize,
breaklines,
]{python}
import numpy as np
import matplotlib.pyplot as plt
from sympy.integrals import inverse_laplace_transform, laplace_transform
from sympy.abc import t, s
from sympy.functions import exp, Heaviside
from sympy import lambdify
# input function
u = lambda t: 20 + 5 * np.exp(-t)
# inverse laplace transform Y(s)
y_expr = inverse_laplace_transform(5 / (s + 1) / (s/6 + 1), s, t)
# convert sympy expression y(t) into lambda func
y = lambdify(t, y_expr, 'numpy')
# lambdify won't work on Heaviside() with sympy version <1.11
t_range_min = np.linspace(0, 6, 100)
# plotting
plt.plot(t_range_min, y(t_range_min) + 20) # +20 to add initial condition back
plt.plot(t_range_min, u(t_range_min), '--')
plt.xlabel(r'$t$ (min)')
plt.ylabel(r'$T(t)$')
plt.title(r'Problem 3.12 Response Plot')
plt.legend(['Response', 'Input'])
\end{minted}


\end{enumerate}
\end{document}